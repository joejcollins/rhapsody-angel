% !TeX root = 00Book.tex
\subsection{April: Reduce to Two Colonies}

April is the beginning of the old farming (and tax) year.
Colonies that survived the winter should be building up.
Now is the time to ensure that the year begins with only two colonies.
Two is the minimum number of colonies to keep since
a lost queen in one colony can be replaced using eggs from the other colony.
The intention is to get down to two strong colonies on 18 frames each (9Fx9F).
The apiary should look like this:

\begin{apiary}{Select of removal}
  \path (4,6) pic{roof=feeder};
  \path (4,4) pic{brood=9F};
  \path (4,2) pic{brood=9F};
  \path (4,0) pic{stand};

  \node at (13.5,8) {Selected for Removal};
  \path (12,6) pic{roof=feeder};
  \path (12,4) pic{brood=9F};
  \path (12,2) pic{brood=9F};
  \path (12,0) pic{stand};

  \path (20,6) pic{roof=feeder};
  \path (20,4) pic{brood=9F};
  \path (20,2) pic{brood=9F};
  \path (20,0) pic{stand};
\end{apiary}

\subsubsection{First Inspection}

Confirm queen present and laying by looking for eggs and larvae.
Then select a colony for removal.
To select the colony for removal, if:

\begin{description}
  \item[All three are strong] select for sale based on temperament or the state of the frames.
  \item[One Colony is weak] sell as a nuc.
  \item[Two Colonies are weak] cull one queen and unite the two colonies.
  \item[One Colony is dead] remove and clean the hive, burn any dirty frames.
\end{description}

\subsubsection*{Prepare for Removal}

Ideally all three colonies will be strong
so one can be selected for sale.
Remove one brood box from the colony selected for sale
and reduce the colony to 11 frames.
Take the surplus frames and put them in the other colonies,
adding additional frames so they have full brood boxes.
So the apiary should look like this:

\begin{apiary}{Stuff}
    \path (6,6) pic{roof};
    \path (6,4) pic{brood=11F};
    \path (6,2) pic{brood=11F};
    \path (6,0) pic{stand};

    \node at (13.5,6) {Remove or Sell};
    \path (12,4) pic{roof};
    \path (12,2) pic{brood=11F};
    \path (12,0) pic{stand};

    \path (18,6) pic{roof};
    \path (18,4) pic{brood=11F};
    \path (18,2) pic{brood=11F};
    \path (18,0) pic{stand};
\end{apiary}
