% !TeX root = 00Book.tex
\subsection{December: Oxalic Acid and Candy}

Applying Oxalic Acid around December 21-22 (Winter solstice).
Typically, the Queen will start laying again in early January.

Oxalic acid trickle on the bee hives.
The syrup is 308 ml water and 308 g sugar (1 to 1).
This nets you 500 ml of syrup for the 35 g of oxalic in the API-Bioxal sachet (35g dissolve in 500 ml of syrup) the treatment is 5 ml per occupied frame space (seam).

\begin{apiary}{Add candy block indicator}
    \path (4,6) pic{roof=candy};
    \path (4,4) pic{brood=8F};
    \path (4,2) pic{brood=8F};
    \path (4,0) pic{stand};

    \path (12,6) pic{roof=candy};
    \path (12,4) pic{brood=8F};
    \path (12,2) pic{brood=8F};
    \path (12,0) pic{stand};

    \path (20,6) pic{roof=candy};
    \path (20,4) pic{brood=8F};
    \path (20,2) pic{brood=8F};
    \path (20,0) pic{stand};
\end{apiary}

\subsubsection*{Make Candy}

\begin{enumerate}
  \item Put 500 ml boiling water in pressure cooker.
  \item Pour in 2.5kg granulated sugar.
  \item Heat up while stirring
  \item Boil for 5 minutes.
  \item Cool the pressure cooker in the sink while stirring until white and opaque.
  \item Pour into 3 x 750ml dishes, it will nearly fill them.
\end{enumerate}
