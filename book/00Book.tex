\documentclass{./BeekeepingBook}

\title{Apiary Procedures \the\year{}}
\author{Joe Collins}
\date{}
\setcounter{tocdepth}{3}

\begin{document}
\maketitle
\thispagestyle{empty}

\begin{abstract}

  \begin{description}
    \item[Spring] Start the season with two colonies, sell any others.
    \item[Summer] Raise two queens from those colonies so you have four colonies.
    \item[Autumn] End of the season reduce to three colonies by selling or uniting.
    \item[Winter] Go into winter with three strong colonies. 
  \end{description}

  \vfill
  \rule{8cm}{0.4pt}\\
  Each year:
  \begin{itemize}
    \item Replace queens.
    \item Harvest some honey. 
    \item Renew some frames.
    \item Treat for varroa in summer and winter.
    \item Never have less than two colonies.
  \end{itemize}

\end{abstract}

\clearpage
\tableofcontents
\thispagestyle{empty}

\clearpage
\section{Spring: Equipment, Feeding and Queen Selection}
\begin{description}
  \item[February] Ensure there is enough equipment for 6 colonies.
  \item[March] Early feeding to avoid starvation.
  \item[April] Reduce to two colonies.
\end{description}
\clearpage
% !TeX root = 00Book.tex
\subsection{February: Equipment}

Apiary equipment.

\begin{description}
    \item[8 Stands] 6 for hives in the season, one to swap and one under the supers that are drying.
    \item[6 Floors] all out a the peak of the season.
    \item[6 Crown Boards] all out a the peak of the season.
    \item[6 Roofs] all out a the peak of the season.
    \item[7 Brood Boxes] 6 out being used with and one spare to swap for repairs.
    \item[14] Dummy boards.

    \item[4 Queen Excluders] 2 on the production hives and 2 when the old colonies expand.
\end{description}

\subsubsection{Hive Stand}

\emph{TODO}

\subsubsection{Varroa Floor}

With slide and replaceable mesh floor.

\includegraphics[width=0.9\textwidth]{../assets/varroa_floor.png}

\subsubsection{Entrance Block}

\emph{TODO}

\subsubsection{Dummy Board}

\emph{TODO}

\subsubsection{Crown Board}

\emph{TODO}

\subsubsection{Roof}

With space for a rapid feeder.

\emph{TODO}

\subsubsection{Varroa Slide}

\emph{TODO}


\clearpage
% !TeX root = 00Book.tex
\subsection{March: Early Feeding}

The most common reason for colony loss is starvation in March before enough feed is available.
Therefore provide feed anyway.
Allows raising of strong colony for queen raising and the early honey.
Feed early, once the bees show through on the candy,
assume that are in need of feeding.

\begin{apiary}{Feed Anyway}
    \path (4,6) pic{roof=candy};
    \path (4,4) pic{brood=8F};
    \path (4,2) pic{brood=8F};
    \path (4,0) pic{stand};

    \path (12,6) pic{roof=candy};
    \path (12,4) pic{brood=8F};
    \path (12,2) pic{brood=8F};
    \path (12,0) pic{stand};

    \path (20,6) pic{roof=feeder};
    \path (20,4) pic{brood=8F};
    \path (20,2) pic{brood=8F};
    \path (20,0) pic{stand};
\end{apiary}
\clearpage
% !TeX root = 00Book.tex
\subsection{April: Reduce to Two Colonies}

April is the beginning of the old farming (and tax) year.
Colonies that survived the winter should be building up.
Now is the time to ensure that the year begins with only two colonies.
Two is the minimum number of colonies to keep since
a lost queen in one colony can be replaced using eggs from the other colony.
The intention is to get down to two strong colonies on 18 frames each (9Fx9F).
The apiary should look like this:

\begin{apiary}{Select for removal}
  \path (4,6) pic{roof=feeder};
  \path (4,4) pic{brood=9F};
  \path (4,2) pic{brood=9F};
  \path (4,0) pic{stand};

  \node at (13.5,8) {Selected for Removal};
  \path (12,6) pic{roof=feeder};
  \path (12,4) pic{brood=9F};
  \path (12,2) pic{brood=9F};
  \path (12,0) pic{stand};

  \path (20,6) pic{roof=feeder};
  \path (20,4) pic{brood=9F};
  \path (20,2) pic{brood=9F};
  \path (20,0) pic{stand};
\end{apiary}

\subsubsection{First Inspection}

Confirm queen present and laying by looking for eggs and larvae.
Then select a colony for removal.
To select the colony for removal, if:

\begin{description}
  \item[All three are strong] select for sale based on temperament or the state of the frames.
  \item[One Colony is weak] sell as a nuc.
  \item[Two Colonies are weak] cull one queen and unite the two colonies.
  \item[One Colony is dead] remove and clean the hive, burn any dirty frames.
\end{description}

\subsubsection*{Prepare for Removal}

Ideally all three colonies will be strong
so one can be selected for sale.
Remove one brood box from the colony selected for sale
and reduce the colony to 11 frames.
Take the surplus frames and put them in the other colonies,
adding additional frames so they have full brood boxes.
So the apiary should look like this:

\begin{apiary}{Stuff}
    \path (6,6) pic{roof};
    \path (6,4) pic{brood=11F};
    \path (6,2) pic{brood=11F};
    \path (6,0) pic{stand};

    \node at (13.5,6) {Remove or Sell};
    \path (12,4) pic{roof};
    \path (12,2) pic{brood=11F};
    \path (12,0) pic{stand};

    \path (18,6) pic{roof};
    \path (18,4) pic{brood=11F};
    \path (18,2) pic{brood=11F};
    \path (18,0) pic{stand};
\end{apiary}


\clearpage
\section{Summer: Production of Queens and Honey}
\begin{description}
  \item[May] Split the colonies to raise two new queens and early honey production.
  \item[June] Confirm the two new queens and collect swarms.
  \item[July] Swarm management and late honey production.
\end{description}
\clearpage
% !TeX root = 00Book.tex
\subsection{May: Increase to Four Colonies}

The colonies will probably be in able to make queens in May.
This provides an opportunity to raise two new queens.
The intention is to 
begin the month with two colonies on 18 frames each (9F x 9F)
and 
end the month (or June) with four colonies.  
Two new colonies on 11 frames each (11F)
and 
the two original colonies on 5 or more frames each.\par

\begin{apiary}{Begin with two colonies (and two empty hives)}
    \path (0,0) pic{stand};
    
    \path (4,6) pic{roof};
    \path (4,4) pic{brood=9F};
    \path (4,2) pic{brood=9F};
    \path (4,0) pic{stand};
    
    \path (8,4) pic{roof};
    \path (8,2) pic{brood=2D};
    \path (8,0) pic{stand};

    \path (16,0) pic{stand};
    
    \path (20,6) pic{roof};
    \path (20,4) pic{brood=9F};
    \path (20,2) pic{brood=9F};
    \path (20,0) pic{stand};
    
    \path (24,4) pic{roof};
    \path (24,2) pic{brood=2D};
    \path (24,0) pic{stand};
\end{apiary}

Beside each of the colonies is an empty hive containing two dummy boards.
When a new queen is being raised, the old queen will be removed to the empty hive
and
a new queen will be raised in the original hive.
There is an empty hive stand to the other side of each colony
to allow flying bees to be bled off from the old queen's colony
by swapping the hive to the other side of the new queen.

\subsubsection{Inspect for Queen Raising}

The goal of the inspection is to determine if the colony is considering reproducing,
(not necessarily that full scale swarm preparations are being made).
So we are looking for queen cells with eggs or grubs.

\begin{figure}[H]
\centering
\begin{tikzpicture}
    \color{white}
    \draw (0, 0) node {\includegraphics[width=0.9\textwidth]{../assets/05MayInspection.jpg}};
    \draw (0, 2) node {Brood Boxes};
    \draw (-3, 0) node {Supers};
    \draw (3, 0.5) node {Spare Brood Box};
\end{tikzpicture}
\caption{Inspect with supers in front of brood, spare brood box to one side}%
\end{figure}

\begin{description}
  \item [Inspect on a 5 to 6 day interval] to give a margin for error.
    From egg to sealed queen cell takes 8 days, so it is possible to inspect on an 8 day interval and many beekeepers use a 7 day interval.
    However, planned inspections can get delayed, maybe rushed or less than through.
    Reducing the inspection interval to 5 to 6 days gives a greater latitude for errors
    in the event that inspections are delayed by weather or unforeseen circumstances.
  \item [If you see the queen remove that frame] to the spare brood box even if no swarm preparations are apparent.
    Swarm preparations may not be apparent until later in the inspection.
    Putting the queen aside on a single frame in separate box ensures that she available should it be necessary to split the colony.
    If no swarm preparations are apparent the frame with the queen on can be replaced at the end of the inspection.
\end{description}

\subsubsection{Raise a New Queen}

History as a guide to tell you when this might happen.
If you see queen cells with eggs,
then you are good to go.
We are not trying to head off a swarm,
but to breed another queen.

\begin{description}
  \item[Remove queen] If you see queen cells with eggs, remove the queen to one side.
    Re-stack the brood boxes.
    Shake in two frames of bees.
    If there are any sealed queen cells, cull them so that none of the remaining queens will emerge during the next week.
  \item[Cull queen cells] 1 week later cull all cells but one.
    There are probably two because you might have missed one, so there is no need to deliberately leave two.
    Choose a good looking one,
    that doesn't look like it will get damaged by your interference.
  \item[Check for laying] 6 weeks after the queen was removed check for eggs.
    If no eggs add a frame of eggs from the original queen
    and start again. 
\end{description}

Take out the queen to one side along with 7 of the least brood laden frames.  
Add 4 new frames.
Put in a brood box with two dummy boards.
We want the brood to hatch in the production hive.
The queen to one side will be the feeder hive.

The other 11 frames with the queenless hive

\subsubsection*{Step 1: Remove queen}

As soon as occupied queen cells are discovered (eggs or grubs)

Not trying to work out if they are going to swarm.

\begin{description}
  \item[Find queen and remove her], on a frame of (mostly) sealed brood + bees. Remove any queen cells from this frame after checking that there are others in the hive.
  If you can't find her  then guess, she is probably in the middle of the brood nest.
  Then close up the hives and listen.  The noisiest hive is probably queenless.
  Wait and come back in 3 days and look for eggs.
  Put old queen on a stand right beside the stand with the new queen.
  So if there is a problem the two colonies can be united.
  \item[Queen excluder underneath] to prevent swarming (remove after a week)
  \item[Add four other frames] the ones bearing the least brood so the queen has space to lay.
  These will also be the frames with the most stores which could be useful since most of the flying bees will be lost.
  \item[Remove all queen cells] sealed or otherwise.  There shouldn't be any but sometimes you are late and there are sealed cells.
  \item[Shake in three frames of house bees] because most of the flying bees will be lost and these can be promoted to 
  \item[Add three new brood frames] bringing it up to eight frames.  The winter configuration is 8 on 8.
  
  
Put frame + queen in nucleus.
Add a second frame of mostly sealed brood, if wished + a frame of food + another l or 2
frames of comb (preferably) or foundation.
Shake in sufficient young worker bees to ensure that there are enough to cover the brood.
Close up the nucleus. Put green grass in the entrance if it is to remain in the same apiary.
Check through the parent colony.  Mark frames containing 2 or 3 good, unsealed, queen cells with a drawing pin.
Close up the frames in the brood box and till the remaining space with frames of comb or foundation.
  Remove any sealed queen cells (Although, to use this method, the old queen must still be present so there should not be any.)
  \item[Add 4 new frames and feed] 1:2 syrup to draw out the frames
\end{description}

\subsubsection*{Step 2: Remove the Queen Excluder}

After 3 days remove the queen excluder from underneath the original queen,
since the chance of swarming should be reduced now.

\subsubsection*{Step 3: Cull queen cells}

At 7 days after the queen was removed.
There are now no eggs or grubs that can be made into a queen, 
so if the queen cells are culled no more can be created.
Only do this once to minimize disturbance,
if you do it too early they will just make more cells and you will have to do it again.

\begin{description}
  \item[Remove any emergency queen cells] Be meticulous.  Go through it twice.  Brush and smoke, don't shake.
  Go through parent colony and remove any emergency queen cells.
  \item[Keep one (only one) queen cell]
  Select l of the cells previously marked and remove the others.
  Some authorities suggest leaving two, incase one is a dud.
  More than likely you will get a swarm.
  In the event that the one left is a dud the old queen is still available and laying to provide eggs for an emergency queens,
  or to unite back with the colony.
  \item[Add four frames for old queen] The old colony should be building up, add 4 more frames and remove the dummy boards to bring it up to 11 frames.
  \item[Move the old queens colony] to the other side to bleed off bees into the new queens colony.
  \item[Leave for five weeks] looking isn't informative, adding eggs will cause a reset and promote further swarming.
  Don't wait long enough and you might miss the new queen, wait too long and there is a chance of laying worker.
  Five weeks is enough to allow the new queen to come into lay.  Longer than this typically ends with a laying worker.
  But is it a compromise.
\end{description}

Advantages of the method
colony remains strong throughout.
Old queen is kept safe and is available if the new queen does not succeed.
The old queen in the nucleus quickly comes back into lay and her brood can be put back into the parent colony.
The method involves minimum time and lifting.
The nucleus is available to use for other procedures later, or can be united back to the original colony.

\subsubsection*{Step 4: Be Ready for Swarms After Culling}

In the subsequent week be alert for swarms,
If you may have missed a queen cell there are a lot of young bees
so any new queen is like to swarm rather than stay and fight it out.

\subsubsection*{Step 5: Check the Original Queen for Excess Numbers}

Continue inspecting the original queen once a week.
The original queen which was moved to one side may expand quite quickly.
If it does bleed off the flying bees by moving it to the other side of the colony with the queen cells.
This ensures that the original queen is less likely to swarm,
and the the colony with the queen cells has excess flying bees to collect honey.

\subsubsection*{Step 6: Check for a Laying Queen}

At 42 days after the queen was removed
(35 days after the queen cells were culled)
check for a laying queen.

\begin{description}
  \item [Good brood pattern] look for the new queen and mark her.
  \item [Eggs at the bottom of cells] you don't know if it is a good queen or a drone layer.
    Wait for 7 days then recheck, drone laying might be apparent at this stage but not over developed.
    Be cautious identifying a drone layer, 
    a recently mated queen can have an uneven pattern that looks a bit like a drone layer.
    If it does turn out to be drone brood, then shake out as below.
  \item[Multiple eggs in cells, eggs on the sides of cells or drone brood only]
    then there is a drone laying queen or a laying worker.
    Either way take the colony 50 metres away and shake all the bees off the brood frames.
    Place the frames back in the original location with a frame of eggs from another colony.
    The flying bees will return to the same location and hopefully be able to raise a queen.
    Keep adding frames of eggs until they raise a queen.
  \item [If there is no sign of eggs or brood] then they state of the colony isn't certain.
    However at this late stage, 9 times out of 10 this will become a laying worker
    so act like it is and shake out the bees as above.
    Shaking out the bees makes certain that the flying bees are subject only to pheromones from the frame of eggs
    and you are unconfused about the state of the colony.
\end{description}

\clearpage
% !TeX root = 00Book.tex
\subsection{June: Production and Swarming}

\subsubsection{Early Honey}

\begin{itemize}
    \item Probably oil seed rape honey so prone to crystallizing in the frames.
    \item Take off early before capped
    \item Dry in airing cupboard.
    \item Extract as soon as you can.
    \item If crystallized melt in bain marie.
\end{itemize}

\subsubsection{Drying Set Up}

\emph{TODO}

\subsubsection{Extraction}

\emph{TODO}

\subsubsection{Wax Recovery}

\emph{TODO}


\subsection{Swarm Collection}

\emph{TODO}


\clearpage
% !TeX root = 00Book.tex
\subsection{July: Honey Production}

Drying with fan in airing cupboard.

9 frames to a super.

To prevent it fermenting and spoiling, honey should have a water content of 17-19\%.
The Honey (England) Regulations 2015 sets out various definitions of what can legally be classified as honey.

\begin{itemize}
    \item Honey must have a moisture content of 20\% or less to be sold as honey;
    \item Bakers' honey lies between 20\% and a maximum of 23\%.
\end{itemize}

The Regulations also specify moisture contents for honey derived from specific plants.


\clearpage
\section{Autumn: Feeding for Winter}
\clearpage
% !TeX root = 00Book.tex
\subsection{August: Honey Crop and Varroa Treatment}

\subsubsection{Honey}

Taking off the honey before varroa treatment.
Canadian rhombus clearer board takes 2 days to clear of bees.

Drying the honey,
in the airing cupboard.

\subsubsection{Varroa Treatment}

No point in judging the number of varroa.
Just treat.
Apiguard or MAQS
On of before 12 kg to ensure that the temperature is h

\subsubsection{Feeding}

It is
generally considered that a honey bee colony requires about 20 – 30 kg of honey to
safely feed it through the winter. 
Feed about 16 kg of sugar per hive.
A brood frame can carry about 2.2 kg of honey.  
So the carrying capacity of 16 frames is about 35 kg.
The brood will take up space.
So for this reason you need at least 16 frames to go through winter.
A single brood is too small but a double brood is too much.
Brood and a half it about right but it is too compact
and you have a mix of frame sizes.

 So a total of 48 kg of sugar is required for all three hives.
 If you can get it at \pounds 1 per kg this is only £50,
 so it is hardly worth bothering to heft the hives or weigh them.
 Just feed it all or until they stop.
 
 \subsubsection{Swarming}

 If it happens it is too late, therefore:

 \begin{description}
     \item [Queen to one side] Split
     \item [Queen excluder underneath] to prevent swarming (remove after a week)
     \item [Unite] again don't try to produce a new queen.
 \end{description}

\clearpage
% !TeX root = 00Book.tex
\subsection{September - Three colonies for winter}

\subsubsection{Cull and Unite}

\emph{TODO}

\subsubsection{Sale}
\clearpage
\input{10October}

\clearpage
\section{Winter: Hive Maintenance}
\clearpage
\input{11November}
\clearpage
% !TeX root = 00Book.tex
\subsection{December: Oxalic Acid and Candy}

Applying Oxalic Acid around December 21-22 (Winter solstice).
Typically, the Queen will start laying again in early January.

Oxalic acid trickle on the bee hives.
The syrup is 308 ml water and 308 g sugar (1 to 1).
This nets you 500 ml of syrup for the 35 g of oxalic in the API-Bioxal sachet (35g dissolve in 500 ml of syrup) the treatment is 5 ml per occupied frame space (seam).

\begin{apiary}{Add candy block indicator}
    \path (4,6) pic{roof=candy};
    \path (4,4) pic{brood=8F};
    \path (4,2) pic{brood=8F};
    \path (4,0) pic{stand};

    \path (12,6) pic{roof=candy};
    \path (12,4) pic{brood=8F};
    \path (12,2) pic{brood=8F};
    \path (12,0) pic{stand};

    \path (20,6) pic{roof=candy};
    \path (20,4) pic{brood=8F};
    \path (20,2) pic{brood=8F};
    \path (20,0) pic{stand};
\end{apiary}

\subsubsection*{Make Candy}

\begin{enumerate}
  \item Put 500 ml boiling water in pressure cooker.
  \item Pour in 2.5kg granulated sugar.
  \item Heat up while stirring
  \item Boil for 5 minutes.
  \item Cool the pressure cooker in the sink while stirring until white and opaque.
  \item Pour into 3 x 750ml dishes, it will nearly fill them.
\end{enumerate}

\clearpage
% !TeX root = 00Book.tex
\subsection{January: Purchasing in the winter sale}

Minimum consumables.

\begin{description}
    \item[22 Brood Frames] because you will sell two complete colonies.
      You might get rid of between 22 and 44 frames so get a pack of 50.
    \item[22 Sheets of Wired Brood Wax] for above.
    \item[22 Red End Spacers] for the North and West ends for above.
    \item[22 Green End Spacers] for the South and East ends for above.
    \item[Apiguard or MAQS for 3 hives] to treat at the end of the season.
      For at least three hives but probably four so a pack of 10 Apiguard trays.
    \item[Oxalic acid for 3 hives] to treat in the middle of winter.
    \item[45 kg Sugar] to feed at the end of the season.
    \item[2.5 kg Sugar] for candy blocks.
    \item[3 x 750 ml take away dishes] for candy blocks.
\end{description}



\clearpage
\section{Notes}

\subsection{Beekeeper Equipment}

\begin{description}
  \item[Suit]  
  \item[Rubber Gloves] 
  \item[Wellingtons] 
  \item[Smoker] 
  \item[Hive Tool] 
  \item[Marking Pen]
  \item[Crown of Thorns]
  \item[Pins] to mark frames that have queen cells on.
  \item[Cocktail Sticks] to extract larvae for examination.
  \item[2 Hive Cover Cloths] to cover the first brood box when you are looking in the second.
  \item[2 Cargo Straps] to help with moving hives. 
  \item[Sheets of Newspaper] to wrap culled drone comb in.
\end{description}

\subsection{Apiary Diagram}
\begin{apiary}{Reference Diagram for Notes}
    \path (0,7.5) pic{roof=feeder};
    \path (0,6)  pic{super=super};
    \path (0,4)  pic{brood=8F};
    \path (0,2)  pic{brood=8F};
    \path (0,0)  pic{stand};
    
    \path (4,7.5) pic{roof=candy};
    \path (4,6)  pic{super=super};
    \path (4,4)  pic{brood=8F};
    \path (4,2)  pic{brood=8F};
    \path (4,0)  pic{stand};

    \path (8,7.5) pic{roof=candy};
    \path (8,6) pic{super=super};
    \path (8,4) pic{brood=8F};
    \path (8,2) pic{brood=8F};
    \path (8,0) pic{stand};

    \path (12,7.5) pic{roof=candy};
    \path (12,6) pic{super=super};
    \path (12,4) pic{brood=8F};
    \path (12,2) pic{brood=8F};
    \path (12,0) pic{stand};

    \path (16,7.5) pic{roof=candy};
    \path (16,6) pic{super=super};
    \path (16,4) pic{brood=8F};
    \path (16,2) pic{brood=8F};
    \path (16,0) pic{stand};

    \path (20,7.5) pic{roof=candy};
    \path (20,6) pic{super=super};
    \path (20,4) pic{brood=8F};
    \path (20,2) pic{brood=8F};
    \path (20,0) pic{stand};
    
    \path (24,7.5) pic{roof=candy};
    \path (24,6) pic{super=super};
    \path (24,4) pic{brood=8F};
    \path (24,2) pic{brood=8F};
    \path (24,0) pic{stand};
\end{apiary}

\end{document}
