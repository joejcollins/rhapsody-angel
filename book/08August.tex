% !TeX root = 00Book.tex
\subsection{August: Honey Crop and Varroa Treatment}

\subsubsection{Honey}

Taking off the honey before varroa treatment.
Canadian rhombus clearer board takes 2 days to clear of bees.

Drying the honey,
in the airing cupboard.

\subsubsection{Varroa Treatment}

No point in judging the number of varroa.
Just treat.
Apiguard or MAQS
On of before 12 kg to ensure that the temperature is h

\subsubsection{Feeding}

It is
generally considered that a honey bee colony requires about 20 – 30 kg of honey to
safely feed it through the winter. 
Feed 14 kg per hive
which comes up to about 16 kg when stored in the hive.

A brood frame can carry about 2.2 kg of honey.  
So the carrying capacity of 16 frames is about 35 kg.
The brood will take up space.
So for this reason you need at least 16 frames to go through winter.
A single brood is too small but a double brood is too much.
Brood and a half it about right but it is too compact
and you have a mix of frame sizes.

 So a total of 42 kg of sugar is required for all three hives.
 If you can get it at 50p per kg this is only £21,
 so it is hardly worth bothering to heft the hives or weigh them.
 Just feed it all or until they stop.
 
 \subsubsection{Swarming}

 If it happens it is too late, therefore:

 \begin{description}
     \item [Queen to one side] Split
     \item [Queen excluder underneath] to prevent swarming (remove after a week)
     \item [Unite] again don't try to produce a new queen.
 \end{description}
