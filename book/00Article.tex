\documentclass{article}
\usepackage[UKenglish]{babel}% http://ctan.org/pkg/babel
\usepackage[UKenglish]{isodate}% http://ctan.org/pkg/isodate

\title{Small Apiary Management with \textbf{2-4-3} }
\author{Joe J Collins}

\date{\today}

\begin{document}

\maketitle

\begin{abstract}
Never have less than two colonies with laying queens,
because if you have problems with one you can often
fix it with eggs and brood from the other.
So start the season in spring with \textbf{2} colonies,
expand to \textbf{4} (or maybe 6) colonies during the summer.
Then select the best queens
and contract to \textbf{3} colonies in autumn ready for the winter.
The approach means that you are able to produce honey,
breed and select new queens and
are able to recover is something goes wrong.
\end{abstract}

\section{April}

If you can came through the winter with all 3 colonies intact,
one needs to be passed on to another beekeeper who lost a colony.
If one of the three died then you are good to start the season.
Feed the colonies to ensure that they are strong and can make new queens.

\section{May}

If the 2 colonies are strong and healthy they will make preparations to swarm.
When they do take out the laying queens and put them to one side.

Swarming and expand to 4

\section{June}

\end{document}

