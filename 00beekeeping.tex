\documentclass{./BeekeepingBook}

\title{324 Apriary Management}
\author{Joe J Collins}
\date{\DTMnow}
 
\begin{document}
 
\maketitle

\section{Keeping Bees}

Doctrine of planned action.

Minimal equipment



\section{January - Oxalic Acid and Candy}

\begin{apiary}{Add candy block indicator}
    \path (4,6) pic{roof=candy};
    \path (4,4) pic{brood=8F};
    \path (4,2) pic{brood=8F};
    \path (4,0) pic{stand};

    \path (12,6) pic{roof=candy};
    \path (12,4) pic{brood=8F};
    \path (12,2) pic{brood=8F};
    \path (12,0) pic{stand};

    \path (20,6) pic{roof=candy};
    \path (20,4) pic{brood=8F};
    \path (20,2) pic{brood=8F};
    \path (20,0) pic{stand};
\end{apiary}

\section{March - Feeding}

The most common reason is starvation.

\begin{apiary}{Feed if necessary}
    \path (4,6) pic{roof=feeder};
    \path (4,4) pic{brood=8F};
    \path (4,2) pic{brood=8F};
    \path (4,0) pic{stand};

    \path (12,6) pic{roof=feeder};
    \path (12,4) pic{brood=8F};
    \path (12,2) pic{brood=8F};
    \path (12,0) pic{stand};

    \path (20,6) pic{roof=feeder};
    \path (20,4) pic{brood=8F};
    \path (20,2) pic{brood=8F};
    \path (20,0) pic{stand};
\end{apiary}

\section{April - Two colonies to begin the season}

\subsection{1st Inspection - Select for Sale}

\begin{enumerate}
	\item{Confirm queen present and laying}
	\item{If all three are still 
	
	Select for sale based on temperament, size or the state of the frames.
	One of the colonies has to go.}
\end{enumerate}

\begin{apiary}{Select of removal}
    \path (4,6) pic{roof=feeder};
    \path (4,4) pic{brood=8F};
    \path (4,2) pic{brood=8F};
    \path (4,0) pic{stand};

    \node at (13.5,8) {Selected for Removal};
    \path (12,6) pic{roof=feeder};
    \path (12,4) pic{brood=8F};
    \path (12,2) pic{brood=8F};
    \path (12,0) pic{stand};

    \path (20,6) pic{roof=feeder};
    \path (20,4) pic{brood=8F};
    \path (20,2) pic{brood=8F};
    \path (20,0) pic{stand};
\end{apiary}

Shuffle for sale or give away.


\begin{apiary}{Stuff}

    \path (4,6) pic{roof};
    \path (4,4) pic{brood=9F};
    \path (4,2) pic{brood=9F};
    \path (4,0) pic{stand};

    \node at (13.5,6) {Remove or Sell};
    \path (12,4) pic{roof};
    \path (12,2) pic{brood=11F};
    \path (12,0) pic{stand};


    \path (20,6) pic{roof};
    \path (20,4) pic{brood=9F};
    \path (20,2) pic{brood=9F};
    \path (20,0) pic{stand};
\end{apiary}

The colony selected for sale has 16 frames (8F x 8F).
Select 11 of the frames for sale.
And put in one box.
Shake off the other 5 frames and put 4 them in the other two colonies so the are now 18 frames (9F x 9F) each.
Stash or trash the 5 th frame.



\section{May - Increase to 4 colonies}

At this point both colonies are on double brood of 18 frames (9F x 9F)


\begin{apiary}{Stuff}
    \path (4,6) pic{roof};
    \path (4,4)  pic{brood=8F};
    \path (4,2)  pic{brood=8F};
    \path (4,0)  pic{stand};

    \path (12,6) pic{roof};
    \path (12,4)  pic{brood=8F};
    \path (12,2)  pic{brood=8F};
    \path (12,0)  pic{stand};

    \path (20,6) pic{roof};
    \path (20,4)  pic{brood=8F};
    \path (20,2)  pic{brood=8F};
    \path (20,0)  pic{stand};
\end{apiary}



\subsection{Swarm inspections}

5 to 6 day interval

Strategy

Find Q first,
Put aside you are doing Celia
Then find QCs and mark
Then shuffle frames

Find QC first
Put aside, you are going artificial swarm
Mark QC
Don’t find Q then you are guessing
 
When you do an artificial swarm transfer a single frame of brood.  
Not to “keep her there” but so you can confirm that the queen is still present because she will most likely be on this frame.


\subsection{Swarm Control - Remove the Queen}

\begin{table}[H]%
\begin{center}
\begin{tabular}{lll}
{\bf Day} & {\bf Queen} & {\bf Beekeeper} \\
1 & \rdelim\}{3}{3mm}[Egg] & $\leftarrow$ First day to remove queen \tabularnewline
2 \tabularnewline
3 & \multirow{2}{*}{\quad $\leftarrow$ Hatching} &  \tabularnewline
\cline{1-1}
4 & \rdelim\}{5}{3mm}[Larva] \tabularnewline
5 &  & \\
6 &  & \\
7 &  & \\
8 & \multirow{2}{*}{\quad $\leftarrow$ Sealed} & $\leftarrow$ Last day to remove queen \\
\cline{1-1}
7 & \rdelim\}{8}{3mm}[Pupa] & $\leftarrow$ First day to cull queen cells \\
8 &  & \\
9 &  & \\
10 &  & \\
11 &  & \\
12 &  & \\
13 &  & $\leftarrow$ Last day to cull queen cells \\
16 & \multirow{2}{*}{\quad $\leftarrow$ Emerging}  & \\
\cline{1-1}
15 &  \rdelim\}{7}{3mm}[Maturing] \\
16 &  & \\
17 &  & \\
18 &  & \\
19 &  & \\
20 &  & \\
21 &  & \\
22 &  & \\
23 &  & \\
\hline
24 & Mating & Hope for warm weather \\
\end{tabular}
\caption{Queen Development}%
\end{center}
\end{table}

Take out the queen to one side along with 7 of the least brood laden frames.  
Put in a brood box with two dummy boards.
We want the brood to hatch in the production hive.
The queen to one side will be the feeder hive.

The other 11 frame





Step 1: As soon as occupied queen cells are discovered.
Find queen and remove her, on a frame of (mostly) sealed brood + bees. Remove any queen cells from this frame after checking that there are others in the hive.
Put frame + queen in nucleus.
Add a second frame of mostly sealed brood, if wished + a frame of food + another l or 2
frames of comb (preferably) or foundation.
Shake in sufficient young worker bees to ensure that there are enough to cover the brood.
Close up the nucleus. Put green grass in the entrance if it is to remain in the same apiary.
Check through the parent colony.  Mark frames containing 2 or 3 good, unsealed, queen cells with a drawing pin.
Close up the frames in the brood box and till the remaining space with frames of comb or foundation.
Remove any sealed queen cells (Although, to use this method, the old queen must still be present so there should not be any.)
 
Step 2: 1 week later.
Go through parent colony and remove any emergency queen cells.  Best to to this three times so every other day.  The last check being a week
Select l of the cells previously marked and remove the others.
Close up the hive and leave strictly alone until queen is mated and laying.
Advantages of the method
Colony remains strong throughout.
Old queen is kept safe and is available if the new queen does not succeed.
The old queen in the nucleus quickly comes back into lay and her brood can be put back into the parent colony.
The method involves minimum time and lifting.
The nucleus is available to use for other procedures later, or can be united back to the original colony.
Disadvantages
You have to be able to find the queen easily.
The nucleus may grow very quickly, so monitor it carefully.
 
Notes from Celia Davis : SBKA Meeting March 12th 2008




\subsection{Artificial Swarm - Remove the Brood}

\subsection{Artificial Swarm - Combo}



\section{June - Production and Swarming}

\subsection{Inhibit Swarming - Brood Removal}

\subsection{Inhibit Swarming - Moving Flying Bees}

\subsection{Swarming - Clipped Queen}

\subsection{Swarming - Unclipped Queen}

\subsection{Introducing a Queen}
 
 \begin{enumerate}
	\item{Best acceptance between 6 hours and 12 hours queenless or after 9 days.}
\end{enumerate}
 
 
 
\section{July - Honey Production}


 
\section{August - Honey Crop and Varroa Treatment}

Taking off the honey.

Marking and Clipping Queens


\section{September - Three colonies for winter}

\subsection{Cull and Unite}


\subsection{Sale}

\section{December - Varroa Treatment and Candy}

\subsection{Make Candy}




\section{March}{Starvation Risk}


\begin{apiary}{Stuff}
    \path (0,7.5) pic{roof=feeder};
    \path (0,6)  pic{super=super};
    \path (0,4)  pic{brood=8F};
    \path (0,2)  pic{brood=8F};
    \path (0,0)  pic{stand};
    
    \path (4,7.5) pic{roof=candy};
    \path (4,6)  pic{super=super};
    \path (4,4)  pic{brood=8F};
    \path (4,2)  pic{brood=8F};
    \path (4,0)  pic{stand};

    \path (8,7.5) pic{roof=candy};
    \path (8,6)  pic{super=super};
    \path (8,4)  pic{brood=8F};
    \path (8,2)  pic{brood=8F};
    \path (8,0)  pic{stand};

    \path (12,7.5) pic{roof=candy};
    \path (12,6)  pic{super=super};
    \path (12,4)  pic{brood=8F};
    \path (12,2)  pic{brood=8F};
    \path (12,0)  pic{stand};

    \path (16,7.5) pic{roof=candy};
    \path (16,6)  pic{super=super};
    \path (16,4)  pic{brood=8F};
    \path (16,2)  pic{brood=8F};
    \path (16,0)  pic{stand};

    \path (20,7.5) pic{roof=candy};
    \path (20,6)  pic{super=super};
    \path (20,4)  pic{brood=8F};
    \path (20,2)  pic{brood=8F};
    \path (20,0)  pic{stand};
    
    \path (24,7.5) pic{roof=candy};
    \path (24,6)  pic{super=super};
    \path (24,4)  pic{brood=8F};
    \path (24,2)  pic{brood=8F};
    \path (24,0)  pic{stand};
\end{apiary}
 
\end{document}
